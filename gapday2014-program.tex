\documentclass[12pt,a4paper]{article}
\usepackage[left=2cm,right=2cm,top=3cm,bottom=3cm]{geometry}
\pagestyle{empty}

\usepackage[utf8]{inputenc}

%
% custom font
%
\usepackage[full]{textcomp} % to get the right copyright, etc.
\usepackage{fbb} % osf for text, lining for math
\usepackage[scaled=.95]{cabin}
\usepackage[varqu,varl]{inconsolata}% typewriter
\usepackage[libertine,bigdelims,vvarbb]{newtxmath}
\usepackage[cal=boondoxo]{mathalfa}% less slanted than STIX
\usepackage[T1]{fontenc}
\useosf


\usepackage{datetime}
\usdate

\usepackage{amsmath}
\usepackage{amssymb}

\usepackage{enumitem}

\usepackage{xcolor}
\usepackage[colorlinks=true]{hyperref}


\newcommand{\TODO}[1]{\textcolor{red}{*** #1 ***}}

\begin{document}

\begin{center}
{\huge\textbf{Program of the GAP Days 2014, August 25-29}\\[2mm]}
Version from \today\ at \currenttime\\[2mm]
Talks are at \href{https://maps.google.com/maps?q=Pontdriesch+14,+Aachen,+Germany&hl=en&ll=50.778617,6.080579&spn=0.004993,0.008969&sll=37.0625,-95.677068&sspn=50.777825,73.476563&oq=pontdriesch+14+&hnear=Pontdriesch+14,+Mitte+52062+Aachen,+Germany&t=m&z=17}{Pontdriesch 14/16}
in room 008, coding sessions in rooms 003 and 103.
\end{center}

% \vortrag{TIME}{AUTHOR}{TITLE}
\newcommand{\vortrag}[3]{#1 & #2 \\ & \textit{#3} \\}
%\newcommand{\vortrag}[3]{#1 & #2: ``\textit{#3}'' \\}

\newcommand{\newday}[1]{\multicolumn{2}{c}{{\large\textbf{#1}}} \\[1em]}


\begin{tabular}{rp{14.5cm}}
%
\newday{Monday, August 25}
10:00 & Coding session and discussion \\
14:00 & Welcome session (room 008) \\
\vortrag{15:00}{VInay Wagh}{\href{https://github.com/homalg-project/LessGenerators}{LessGenerators} -- finding small generating sets for modules (part of the homalg project)}
\vortrag{15:30}{Martin Bies}{String theory, sheaf cohomology and the homalg package}
\vortrag{16:00}{Johannes Hahn}{Coxeter groups and Kazhdan-Lusztig theory in GAP}
\vortrag{16:30}{Chris Jefferson}{Ferret -- a modern C++ rewrite of Partition Backtracking in GAP}
\vortrag{17:00}{Max Horn}{\href{http://gap-system.github.io/libsing/}{libsing} -- an interface between \href{http://www.singular.uni-kl.de/}{Singular} and GAP}
\vortrag{17:30}{Christof Söger}{\href{https://github.com/fingolfin/NormalizInterface}{NormalizInterface} -- an interface between \href{http://www.math.uos.de/normaliz}{normaliz} and GAP}
%
%
\\
%
%
\newday{Tuesday, August 26}
\vortrag{10:00}{Sebastian Gutsche \& Max Horn}{How to make a GAP package}
\vortrag{16:00}{Pedro A. García-Sánchez
}{New features of the \href{http://www.gap-system.org/Packages/numericalsgps.html}{numericalsgps} package}
\vortrag{16:30}{Manuel Delgado}{\href{http://www.gap-system.org/Packages/intpic.html}{intpic} -- a package for drawing integers, by emphasizing some subsets.}
\vortrag{17:00}{Hebert Perez-Roses}{Graph construction via voltage assignment with GAP}
\vortrag{17:30}{Delaram Kahrobaei}{Conjugacy problem in polycyclic groups in GAP and applications}
%
%
%\\
\end{tabular}

\newpage

\begin{tabular}{rp{14.5cm}}
%
%
\newday{Wednesday, August 27}
\vortrag{10:00}{Reimer Behrends}{HPC-GAP: Design and Implementation of a Concurrency Model for GAP}
\vortrag{16:20}{Markus Pfeiffer}{Two (HPC)GAP infrastructure packages in the making: GAPData and Matrix}
\vortrag{16:50}{Sebastian Gutsche \& Sebastian Posur}{CategoriesForHomalg - A GAP-based meta language for category theory based computations \& ToolsForHomalg - Tools for caching and propagation}
\vortrag{17:30}{Thomas Breuer}{Recent progress concerning the GAP packages AtlasRep, CTblLib, CTBlocks, MFER}

19:00 & Dinner at the ``\href{http://www.labyrinthaachen.de/}{Labyrinth}'' \\
%
%
\\
%
%
\newday{Thursday, August 28}
\vortrag{10:00}{Alexander Konovalov}{Continuous integration, package update mechanism and release management in GAP}
%
14:00 & Group photo (in front of the building) \\ 
15:00 & Open discussion on the GAP decision making process, development model and more \\
%
%
\\
%
%
\newday{Friday, August 29}
10:00 & Open discussion: Your wishes for the future of GAP \\ 
13:30 & Open discussion: Results of the meeting, feedback \\

\end{tabular}

\vfill

\begin{center}
  \url{http://gapdays2014.coxeter.de/}
\end{center}


\pagebreak


% \Abstract{NAME}{AFFIL}{TITLE}
%\newenvironment{Abstract}[3]{\noindent\textbf{#1} (#2) \\ \noindent \textit{#3} \\ \noindent  \par{} \noindent}{\bigskip}
\newenvironment{Abstract}[3]{\begin{itemize}[itemsep=0mm,label={}]
  \item \textbf{#1} (#2)
  \item ``\textit{#3}''
  \item}{\end{itemize}\medskip}


%\begin{center}
{\noindent\huge\textbf{Abstracts of talks and sessions}} \\[1em]
%\end{center}

\begin{Abstract}{Pedro A. García-Sánchez}{Universidad de Granada}{New features of the \texttt{numericalsgps} package}
We will talk about the new functionalities of the development version
of \texttt{numericalsgps}, that will become a new version during the gap
days. We will also review some issues we encountered on the way, specifically
dealing with polynomials and solutions of linear Diophantine equations. For
the functions dealing with polynomials we dealt with the \texttt{singular}
package, for the second we used \texttt{4ti2Interface}. The difficulty
for the avarage user to install and work with these packages prevented us
to include some already implemented functions in the package. Finally,
we will discuss some of he future plans for \texttt{numericalsgps}.
\end{Abstract}


\begin{Abstract}{Sebastian Gutsche \& Max Horn}{TU Kaiserslautern \& JLU Gießen}{How to make a GAP package}

In this hands-on workshop, we will explain the basic requirements for
creating a simple GAP package from scratch. After a brief introduction,
participants can immediately apply this with our help. For this,
participants should bring their laptops and, if present, some code they
want to publish in a package.
%
We also plan to cover more advanced aspects of creating and maintaining
a GAP package.  Which topics are covered in part also depends on requests
by participants. Some possibilities include:
\begin{itemize}[itemsep=0mm]
\item ``Package manuals done right: GAPDoc and AutoDoc'' (this will definitely be covered)
\item Integrating C / C++ code into a GAP package
\item Using GitHub pages as website for your package
\item Automating the package release process with GitHub
\item The importance of package tests and continuous integration
\item Example for automated testing using GitHub and Jenkins
\item ...
\end{itemize}

\end{Abstract}


\begin{Abstract}{Delaram Kahrobaei}{City University of New York}{Conjugacy problem in polycyclic groups in GAP and applications}
Polycyclic Package in GAP has a great computational capacity.
B. Eick and W. Nickel, Polycyclic: Computation with polycyclic groups, a GAP 4 package, \url{http://www.gap-system.org/Packages/polycyclic.html}.

In this talk particularly, I discuss constructing some infinite polycyclic groups and solving the conjugacy problem (both deterministically and heuristically) using GAP Polycyclic package. 

I also will address some cryptographic applications of polycyclic groups.
\end{Abstract}


\begin{Abstract}{Hebert Perez-Roses}{University of Lleida, Spain}{Graph construction via voltage assignment with GAP}
The voltage assignment technique takes a directed "base" graph $B$, and a
group $G$, and constructs another graph $L$ with $|L|=|B||G|$ vertices, where
$|B|$ is the number of vertices of $|B|$. $L$ is usually called the lift of $B$
by $G$, and is a generalization of Cayley graphs. The voltage assignment
technique has been very successful in the construction of large graphs
with small degree and diameter. We are now working on the implementation
of this technique in GAP, and we would like to bring into consideration
of the GAP community the algorithms and data structures used, as well as
to discuss the best alternatives for an efficient implementation.
\end{Abstract}


\begin{Abstract}{Christof Söger}{Universität Osnabrück}{\href{https://github.com/fingolfin/NormalizInterface}{NormalizInterface} -- an interface between \href{http://www.math.uos.de/normaliz}{normaliz} and GAP}
Normaliz is a software for computations with rational cones and affine
monoids. It pursues two main computational goals: finding the Hilbert
basis, a minimal generating system of the monoid of lattice points of a
cone; and counting elements degree-wise in a generating function, the
Hilbert series. As a recent extension, Normaliz can handle unbounded
polyhedra. The Hilbert basis computation can be considered as solving a
linear diophantine system of inhomogeneous equations, inequalities and
congruences.

We are working on a Normaliz interface to GAP. It encapsulates a
libnormaliz cone and gives access to it in the GAP enviroment. In this
way GAP can be used as interactive interface to libnormaliz. We will
show how it can be used currently.
\end{Abstract}


\begin{Abstract}{VInay Wagh}{IIT Guwahati, India}{\href{https://github.com/homalg-project/LessGenerators}{LessGenerators} -- finding small generating sets for modules (part of the homalg project)}
A GAP package called "LessGenerators" has been developed by Mohamed
Barakat and myself, to implement the Quillen-Suslin algorithm in
computer algebra systems SINGULAR and GAP. The package is part of the
homalg project. The aim of this package is to provide a tool for finding
a minimal generating set for a given module. The package provides
universal implementation in the sense of CASs, i.e. it can use any CAS
supported by the homalg project for ring arithmetic.
\end{Abstract}


\end{document}
